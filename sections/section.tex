\section{Introduction}
\label{introduction}

In mathematics and programming problems, we often encounter recursive expressions.
Take the Fibonacci sequence for example, if we compute it directly, it will cause inefficiency, especially in cases that involve multiple iterations or higher-order systems.
As the chart shows below, the computation time will soar up quickly when $N$ is bigger than $40$, which is something we are unwilling to see.

By transforming recursive expressions into matrices form, we can use fast matrix exponentiation to hugely improve the computation efficiency.

\section{Materials and Methods}
\label{materials & methods}
\subsection{Materials}

Pen, paper, computer, C++, brains, curiosity for Linear Algebra.

\subsection{Methods}

From the Rail problem, we derived the recurrence relation and its transition matrix.
Then, we implemented recursive functions, DP, and fast matrix exponentiation to compare their computation times across different values of $N$.

\section{Result}

Belowing are the graphs of computation time vs. $N$ for each methods:
\begin{enumerate}
    \item \textbf{Recursive functions} : Execution time grows exponentially.
    \item \textbf{Time complexity $O(N^2)$ DP (hereafter referred to as DP-1)} : Execution time grows quadratically.
    \item \textbf{Time complexity $O(N)$ DP (hereafter referred to as DP-2)} : Execution time grows linearly.
    \item \textbf{Fast matrix exponentiation} : Execution time is roughly constant.
\end{enumerate}

\section{Discussion}
\label{discussion}

Comparison of the time taken by each method within a specific range of $N$:

Notice that due to the significant differences in execution efficiency among the four methods, it is difficult to compare them directly for the same range of $N$. 
Therefore, we will discuss them in segments based on different ranges of $N$ and select appropriate methods (two or more) for comparison.

\begin{enumerate}
    \item \textbf{When $N \leq 40000$}: 
    
    As shown in the figure below, the time taken by DP-2 grows quadratically, reaching approximately 7000ms at \(N=40000\). In contrast, the time taken by DP-1 and matrix exponentiation methods remains relatively small.
    \item \textbf{When $N \leq 10^7$}:
    
    As shown in the figure below, the time taken by DP-1 grows linearly, reaching approximately 1 second at \(N=4 \times 10^7\). In this range, the matrix exponentiation method remains roughly constant and takes less time than DP-1.

    \item \textbf{When $N \leq 10^{400}$}:
    
    In this very large range of numbers, the time taken by matrix exponentiation grows linearly with \( \log N \), reaching approximately 5000ms at \(N = 10^{400}\).
\end{enumerate}

\section{Conclusion}
\label{conclusion}

\subsection{Efficiency}
\begin{enumerate}
    \item [a.] fast matrix exponentiation > DP > recursion
    \item [b.] As $N$ grows larger, the time gap in cost increases. 
\end{enumerate}

\subsection{Possibility of converting to matrix}
\begin{enumerate}
    \item[a.]$D_n = \sum_{j=1}^k \lambda_j D_{n-j}\text{ where }k,\lambda_1, \lambda_2, \dots, \lambda_k \text{ are consts, then}$ \\
    \\
    $A=\begin{bmatrix}
    \lambda_1 & \lambda_2 & \dots & \lambda_k\\
    1 & 0 & \dots & 0\\
    0 & 1 & \dots & 0\\
    \vdots & \vdots & \ddots  & \vdots \\
    0 & 0 & \dots & 1 
    \end{bmatrix}
    \text{ and }
    \begin{bmatrix}
    D_n \\
    D_{n-1}\\
    \vdots \\
    D_{n-k+1}
    \end{bmatrix}
    =
    A^{n-k}
    \begin{bmatrix}
    D_{k} \\
    D_{k-1}\\
    \vdots \\
    D_{1}
    \end{bmatrix}\\
$
    \\\
    \item[b.] $D_n = \sum_{j=c_1}^{n-c_2} (an+bj+c) D_j$ \\
    \\
    $\rightarrow D_n = \sum_{j=c_1}^{n-c2}(an+c)D_j + b\sum_{j=c_1}^{n-c2}jD_j$\\
    \\
    $\rightarrow D_{n-1} = \sum_{j=c_1}^{n-c2-1}(a(n-1)+c)D_j + b\sum_{j=c_1}^{n-c2-1}jD_j$\\
    \\
    $\rightarrow D_n - D_{n-1} = (an+bn-b\times c_2+c)D_{n-c_2} + a\sum_{j=c_1}^{n-c_2-1}D_j$ \\
    \\
    $\rightarrow D_n - D_{n-1} - (an+bn-b\times c_2+c)D_{n-c_2} = a\sum_{j=c_1}^{n-c_2-1}D_j$\\
    \\
    $\rightarrow D_{n-1} - D_{n-2} - (a(n-1)+b(n-1)-b\times c_2+c)D_{n-c_2-1} = a\sum_{j=c_1}^{n-c_2-2}D_j$\\
    \\
    $\rightarrow D_n - 2D_{n-1} + D_{n-2} - (an+bn-b\times c_2+c)D_{n-c_2} + (an+bn-a-b-b\times c_2+c)D_{n-c_2-1} = aD_{n-c_2-1}$\\
    \\
    $\rightarrow D_n = 2D_{n-1} - D_{n-2} + (an+bn-b\times c_2+c)D_{n-c_2} - (an+bn-2a-b-b\times c_2+c)D_{n-c_2-1}$\\
    \\
    $\text{if } an+bn=0 \Rightarrow a=-b \text{ then }$\\
    \\
    $A=\begin{bmatrix}
    2 & -1 & 0 & \dots & 0 & (-b\times c_2+c) & (2a+b+b\times c_2-c)\\
    1 & 0 & 0 & \dots & 0 & 0 & 0\\
    0 & 1 & 0 & \dots & 0 & 0 & 0\\
    \vdots & \vdots & \vdots & \ddots & \vdots & \vdots & \vdots\\
    0 & 0 & 0 & \dots & 0 & 1 & 0 
    \end{bmatrix}$\\
    \\
    $\text{and }
    \begin{bmatrix}
    D_n \\
    D_{n-1}\\
    \vdots \\
    D_{n-c_2}
    \end{bmatrix}
    =
    A^{n-c_2-1}
    \begin{bmatrix}
    D_{c_2+1} \\
    D_{c_2}\\
    \vdots \\
    D_{1}
    \end{bmatrix}\\$
\end{enumerate}

\section{Reference}